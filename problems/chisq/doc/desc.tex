\documentclass[12pt]{extarticle}
%\usepackage[utf8]{inputenc}
\usepackage{amsmath}
\usepackage{verbatim}
\usepackage{setspace}
\onehalfspace

\newcommand{\pn}{\par\noindent}
\newcommand{\valami}[3]{\vspace{#1cm}\pn {\bf #2} #3}
\newcommand{\feladatp}[6]{
  \vspace{0.1cm}
  \centerline {\bf #1}
  \vspace{0.2cm}
  \pn #2 
  \vspace{0.3cm}
  \pn {\bf input}
%  \pn #3
  #3
  \vspace{0.1cm}
  \pn {\bf output}
  \pn #4
  \vspace{0.2cm}
  \pn {\bf megjegyzés}
  \pn #5
  \vspace{0.3cm}
  \pn {\bf input A}
  \verbatiminput{../io/#6.in}
  \vspace{0.1cm}
  \pn {\bf output A}
  \verbatiminput{../io/#6.out}
}

\newcommand{\feladatpp}[7]{
  \feladatp{#1}{#2}{#3}{#4}{#5}{#6}
  \vspace{0.3cm}
  \pn {\bf input B}
  \verbatiminput{../io/#7.in}
  \vspace{0.1cm}
  \pn {\bf output B}
  \verbatiminput{../io/#7.out}
}


\begin{document}

\feladatpp{chisq}{
\pn Adott $k, \alpha,\ \ p_1,\ldots,p_k$ valségeloszlás és $x_1,\ldots,x_m$ minta  
esetén végezzünk illeszkedés-vizsgálatot. Paraméterek: $k$ egy pozitív egész szám, jelentése a megfigyelt 
valségi változó értékei az $[1,k]$ intervallumbeli egészek, $p_k$: annak az elméleti (hipotetikus)
valsége hogy a megfigyelt változó $k$-t vesz fel.
A kimenet ($dec\in \{0,1\}$) attól függően, hogy megtartjuk $H_0$-t vagy sem $\alpha$ elsőfajú 
hibát használva.
}{
%input
\pn $k\ \  \alpha$
\pn $p_1\ \ldots\ p_k$
\pn $x_1\ \ldots\ x_m$
}{
%output
\pn $dec$
}{
\pn $2\le m \le 10000$
\pn $2\le k \le 100$
}{1}{3}

\end{document}
}
