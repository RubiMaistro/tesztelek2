\documentclass[12pt]{extarticle}
%\usepackage[utf8]{inputenc}
\usepackage{amsmath}
\usepackage{verbatim}
\usepackage{setspace}
\onehalfspace

\newcommand{\pn}{\par\noindent}
\newcommand{\valami}[3]{\vspace{#1cm}\pn {\bf #2} #3}
\newcommand{\feladatp}[6]{
  \vspace{0.1cm}
  \centerline {\bf #1}
  \vspace{0.2cm}
  \pn #2 
  \vspace{0.3cm}
  \pn {\bf input}
%  \pn #3
  #3
  \vspace{0.1cm}
  \pn {\bf output}
  \pn #4
  \vspace{0.2cm}
  \pn {\bf megjegyzés}
  \pn #5
  \vspace{0.3cm}
  \pn {\bf input A}
  \verbatiminput{../io/#6.in}
  \vspace{0.1cm}
  \pn {\bf output A}
  \verbatiminput{../io/#6.out}
}

\newcommand{\feladatpp}[7]{
  \feladatp{#1}{#2}{#3}{#4}{#5}{#6}
  \vspace{0.3cm}
  \pn {\bf input B}
  \verbatiminput{../io/#7.in}
  \vspace{0.1cm}
  \pn {\bf output B}
  \verbatiminput{../io/#7.out}
}


\begin{document}

\feladatpp{ranvar3}{
\pn Egy $X$ valváltozó az $x_1,\ldots ,x_n$ értékeket $p_1,\ldots.p_n$ valségekkel vesz fel.
Számoljuk az eloszlásfüggvényének az $f_1,\ldots, f_m$ értékeit a 
$q_1,\ldots , q_m$ helyeken.
}{
%input
\pn $n\ \ m$
\pn $x_1 \ldots x_n$
\pn $p_1 \ldots p_n$
\pn $q_1 \ldots q_m$
}{
%output
\pn $f_1 \ldots f_m$
}{
%megjegyzés
\pn $1<n,m<100$
\pn Az $F_X(x)=P(X<x)$ definíciót használjuk.
}{1}{4}

\end{document}
